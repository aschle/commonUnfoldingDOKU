\section{Projektmanagement}
\label{sec:projektmanagement}

Im Folgenden wird die Projektgruppe vorgestellt und die Zusammenarbeit im Team, sowie die dabei entstandenen Herausforderungen und Schwierigkeiten erläutert. Auf die Auswirkungen der Teamarbeit auf den Entwicklungsprozess wird im Anschluss eingegangen. Es wird der geplante, sowie umgesetzte Entwicklungsprozess mit Aufgabenverteilung und Zeitplan vorgestellt.

%%%%%%%%%%%%%%%%%%%%%%%%%%%%%%%%%%%%%%%%%%%%%%%%%%%%%%%%%%%%%%%%%%%%%%%%%%%%%%%
%%%%%%%%%%%%%%%%%%%%%%%%%%%%%%%%%%%%%%%%%%%%%%%%%%%%%%%%%%%%%%%%%%%%%%%%%%%%%%%
%%%%%%%%%%%%%%%%%%%%%%%%%%%%%%%%%%%%%%%%%%%%%%%%%%%%%%%%%%%%%%%%%%%%%%%%%%%%%%%
% TEAMMITGLIEDER %%%%%%%%%%%%%%%%%%%%%%%%%%%%%%%%%%%%%%%%%%%%%%%%%%%%%%%%%%%%%%
%%%%%%%%%%%%%%%%%%%%%%%%%%%%%%%%%%%%%%%%%%%%%%%%%%%%%%%%%%%%%%%%%%%%%%%%%%%%%%%
\subsection{Projektgruppe und Teamarbeit}
\label{subsec:teammitglieder}

Das Softwareprojekt wurde mit vier Teammitgliedern begonnen. Dazu gehören Alexa, Friedrich, Henry und Michael. Kurz vor Abschluss des Semesters ist Michael spontan abgesprungen, was von Beginn an zu erwarten war.

Die Tatsache, dass wir uns alle nicht kannten, weder aus vorangegangenen Kursen noch privat, stellte eine große Herausforderung in der Teamarbeit dar. Die Einschätzung der Zuverlässigkeit und des Engagements, sowie der Programmierkenntnisse war dadurch schwierig. Wir hatten alle unterschiedliche Vorerfahrungen und bevorzugten unterschiedliche Programmiersprachen. Diese Dinge erschwerten zu Beginn des Projektes die Einarbeitungszeit und die Aufgabenverteilung.

Da wir nur vier Leute im Team waren haben wir uns dazu entschieden niemanden zum Projektleiter zu ernennen, jeder sollte gleichermaßen Verantwortung für das Projekt übernehmen.


%%%%%%%%%%%%%%%%%%%%%%%%%%%%%%%%%%%%%%%%%%%%%%%%%%%%%%%%%%%%%%%%%%%%%%%%%%%%%%%
%%%%%%%%%%%%%%%%%%%%%%%%%%%%%%%%%%%%%%%%%%%%%%%%%%%%%%%%%%%%%%%%%%%%%%%%%%%%%%%
%%%%%%%%%%%%%%%%%%%%%%%%%%%%%%%%%%%%%%%%%%%%%%%%%%%%%%%%%%%%%%%%%%%%%%%%%%%%%%%
% TEAMMITGLIEDER %%%%%%%%%%%%%%%%%%%%%%%%%%%%%%%%%%%%%%%%%%%%%%%%%%%%%%%%%%%%%%
%%%%%%%%%%%%%%%%%%%%%%%%%%%%%%%%%%%%%%%%%%%%%%%%%%%%%%%%%%%%%%%%%%%%%%%%%%%%%%%
\subsection{Entwicklungsprozess}
\label{subsec:prozess}

Wir haben uns für das Hosting des Projektes bei Spline\footnote{\url{https://dev.spline.de/trac/CommonUnfold/browser/trunk}} entschieden. Zur Codeversionierung wird dabei SVN verwendet. Zusätzlich wird ein Bug-Tracker\footnote{\url{https://dev.spline.de/trac/CommonUnfold/}} direkt von Spline zur Verfügung gestellt.\\

Als Programmiersprache haben wir uns auf \emph{Python} geeinigt, da der Vorgegebene Prototyp in Python implementiert war. Nur einer aus unserer Projektgruppe, Henry, war sehr sicher in Python, alle anderen mussten sich erst einmal in die neue Programmiersprache einarbeiten. Bevor wir mit der Implementierung anfingen, hatte jeder Zeit sich Einzuarbeiten. Für die Programmierung der grafischen Oberfläche haben wir \emph{TkInter} verwendet, ein Wrapper des Tk-Toolkits für Python.\\

Wir haben uns für kein festes Vorgehensmodell zur Softwareentwicklung, wie beispielsweise Wasserfallmodell, SRUM oder TDD , entschieden. Um die Kommunikation im Team und unseren Prozess zu unterstützen, sowie um eine regelmäßige Arbeitsweise aller zu forcieren haben wir uns für die folgenden Prozesselemente und Regeln entschieden:

  \begin{description}
    \item[Lauffähige Version im Repository] Um ein effektives Arbeiten zu ermöglichen, sollten nur lauffähige Versionen committed werden.
    \item[Wöchentliche Treffen] Bei regelmäßigen, wöchentlichen Treffen (meist freitags 14:00 Uhr) haben wir über den aktuellen Stand und über neue Aufgaben gesprochen. Hier wurden Tickets eingetragen und Aufgaben direkt verteilt.
    \item[Trac und Tickets] Die neuen Aufgaben, welche gemeinsam formuliert und besprochen wurden, wurden im Trac als Ticket gemeinschaftlich eingetragen.
    \item[Definierte Aufgaben pro Person] Jeder konnte sich Aufgaben aus den Tickets auswählen und sich verbindlich zuweisen. Jeder hatte dann für gewöhnlich ein oder zwei Aufgaben \bzw Tickets bis zum nächsten Treffen zu erledigen.
    \item[Aussagekräftige Commits] Aussagekräftige Commit Messages sollten das spätere Dokumentieren vereinfachen.
    \item[Bugs] Gefunden Fehler, während der Implementierung, sollten sofort im Trac eingetragen werden. 
  \end{description}

Das Softwareprojekt wurde innerhalb von 15 Wochen durchgeführt. Einen Zeitplan haben wir nicht erstellt. Jede Woche wurden Aufgaben definiert, welche bis zur darauf folgenden Woche erledigt werden sollten.\\

Hier wäre eine Grafik mit den Tickets offene und geschlossene, zeitlicher Verlauf ganz schön. [XXX-TODO]


%%%%%%%%%%%%%%%%%%%%%%%%%%%%%%%%%%%%%%%%%%%%%%%%%%%%%%%%%%%%%%%%%%%%%%%%%%%%%%%
%%%%%%%%%%%%%%%%%%%%%%%%%%%%%%%%%%%%%%%%%%%%%%%%%%%%%%%%%%%%%%%%%%%%%%%%%%%%%%%
%%%%%%%%%%%%%%%%%%%%%%%%%%%%%%%%%%%%%%%%%%%%%%%%%%%%%%%%%%%%%%%%%%%%%%%%%%%%%%%
% FAZIT %%%%%%%%%%%%%%%%%%%%%%%%%%%%%%%%%%%%%%%%%%%%%%%%%%%%%%%%%%%%%%%%%%%%%%%
%%%%%%%%%%%%%%%%%%%%%%%%%%%%%%%%%%%%%%%%%%%%%%%%%%%%%%%%%%%%%%%%%%%%%%%%%%%%%%%
\subsection{Fazit zum Projektmanagement}
\label{subsec:fazitPM}

Für unser Softwareprojekt war unser Vorgehen als Gruppe nicht sehr optimal, hat aber am Ende zu einem guten und lauffähigen Programm geführt. Trotzdem würden wir als Gruppe in einem neuen Projekt viele Dinge anders gestalten.

Ein erster Punkt wäre die Auswahl der Programmiersprache, wobei wir beim nächsten Mal eine wählen wollen, welche alle Teammitglieder ausreichend gut beherrschen, beispielsweise \emph{Java}. Diese Entscheidung würde unser radikales Vorgehen etwas entradikalisieren. Das erstellen eines Zeitplans mit Meilensteinen und Arbeitspaketen ist ein Muss für das nächste Projekt.
Das Einsetzen eines Teamleiters, welcher nach außenhin die Kommunikation mit dem Professor und anderen Team übernimmt und sich um die Organisation von Teamtreffen kümmert ist eine gute Sache. Um die Arbeitsmoral stets angenehm und positiv zu halten würdem wir Coding Sessions einplanen, wo sich alle Teammitglider an einem Ort zum Programmieren verabreden würden. Regelmäßige Protokolle sollten abwesenden Mitgliedern helfen auf dem aktuellen Stand der Entwicklung zu bleiben und Entscheidungen nachzuvollziehen.
Wir haben darüber nachgedacht ein Vorgehensmodell zur Softwareentwicklung einzusetzen.