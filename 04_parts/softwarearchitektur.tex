\section{Backend und Softwarearchitektur}
\label{sec:backend}

%%%%%%%%%%%%%%%%%%%%%%%%%%%%%%%%%%%%%%%%%%%%%%%%%%%%%%%%%%%%%%%%%%%%%%%%%%%%%%%
%%%%%%%%%%%%%%%%%%%%%%%%%%%%%%%%%%%%%%%%%%%%%%%%%%%%%%%%%%%%%%%%%%%%%%%%%%%%%%%
%%%%%%%%%%%%%%%%%%%%%%%%%%%%%%%%%%%%%%%%%%%%%%%%%%%%%%%%%%%%%%%%%%%%%%%%%%%%%%%
% FUNKTIONALE ANFORDERUNGEN %%%%%%%%%%%%%%%%%%%%%%%%%%%%%%%%%%%%%%%%%%%%%%%%%%%
%%%%%%%%%%%%%%%%%%%%%%%%%%%%%%%%%%%%%%%%%%%%%%%%%%%%%%%%%%%%%%%%%%%%%%%%%%%%%%%
\subsection{Funktionale Anforderungen}
\label{subsec:anforderungen}

Im Fogenden werden wir die Funktionalen Anforderung an unser Programm erläutern. Diese waren zum Teil durch die Aufgabenstellung gegeben, ergaben sich aber auch während der Programmierarbeit.

  \begin{description}
    \item [{Optionale~Verknüpfung~von~Randstücken:}] Wird ein Gitternetz     eines Quaders gezeichnet, so entstehen doppelte Linien, obwohl es dies Kanten nur einmal gibt. Hier galt es eine Regel zu finden um Ungenauigkeiten zu vermeiden.
    \item [{Unterstützung~der~Anfangszuordnung:}] Bevor mit dem Zeichnen begonnen wird, sollen die Startpunkte auf den einzelnen Schachteln ausgewählt werden können.
    \item [{Automatisierung~der~Verarbeitung~verschiedener~Schachteln:}] Es soll durch den Benutzer ausgewählt werden, welche Schachteln er bearbeten
    möchte.
    \item [{Automatisches~Auffüllen:}] Das ist ein wenig komplizierter, wird an einer Stelle etwas weggenommen, weil es zum Beispiel durch eine ander Fläche übermalt wurde, so sollen automatisch die Zwischenräume aufgefüllt werden.
  \end{description}

Wir haben eine Trennung von MVC-Pattern angestrebt. Trennun gvon GUI
und Logik.\\


%%%%%%%%%%%%%%%%%%%%%%%%%%%%%%%%%%%%%%%%%%%%%%%%%%%%%%%%%%%%%%%%%%%%%%%%%%%%%%%
%%%%%%%%%%%%%%%%%%%%%%%%%%%%%%%%%%%%%%%%%%%%%%%%%%%%%%%%%%%%%%%%%%%%%%%%%%%%%%%
%%%%%%%%%%%%%%%%%%%%%%%%%%%%%%%%%%%%%%%%%%%%%%%%%%%%%%%%%%%%%%%%%%%%%%%%%%%%%%%
% TECHNOLOGIEN %%%%%%%%%%%%%%%%%%%%%%%%%%%%%%%%%%%%%%%%%%%%%%%%%%%%%%%%%%%%%%%%
%%%%%%%%%%%%%%%%%%%%%%%%%%%%%%%%%%%%%%%%%%%%%%%%%%%%%%%%%%%%%%%%%%%%%%%%%%%%%%%
\subsection{Verwendete Technologien und Systemvoraussetzungen}
\label{subsec:technologien}


SystemvoraussetzungenCommon unfolding\\
Welches Betreibssystem wird benötigt?\\
Welche Abhängigkeiten sind zu beachten?\\
Wie führe ich das Programm aus?\\


%%%%%%%%%%%%%%%%%%%%%%%%%%%%%%%%%%%%%%%%%%%%%%%%%%%%%%%%%%%%%%%%%%%%%%%%%%%%%%%
%%%%%%%%%%%%%%%%%%%%%%%%%%%%%%%%%%%%%%%%%%%%%%%%%%%%%%%%%%%%%%%%%%%%%%%%%%%%%%%
%%%%%%%%%%%%%%%%%%%%%%%%%%%%%%%%%%%%%%%%%%%%%%%%%%%%%%%%%%%%%%%%%%%%%%%%%%%%%%%
% FUNKTIONSWEISE %%%%%%%%%%%%%%%%%%%%%%%%%%%%%%%%%%%%%%%%%%%%%%%%%%%%%%%%%%%%%%
%%%%%%%%%%%%%%%%%%%%%%%%%%%%%%%%%%%%%%%%%%%%%%%%%%%%%%%%%%%%%%%%%%%%%%%%%%%%%%%
\subsection{Beschreibung der Funktionsweise}
\label{subsec:funktionsweise}

Alle Zeichenvorgänge gehen von einer Benutzereingabe auf der Hauptzeichenfläche aus, auch beim Zeichnen auf den Schachtel-Zeichenflächen wird ein entsprechendes Ereignis auf der Hauptfläche ausgelöst, welches in der draw-Methode des DrawingCanvas-Objektes behandelt wird.\\

Von den x- und y-Koordinaten dieses Ereignisses ausgehend wird von jeder Schachtel-Zeichenfläche die Funktion prepare(x, y) aufgerufen.\\

In den Zeichenflächen werden die Koordinaten berechnet, die, beim aktuellen Zeichenvorgang, dem Punkt auf der Hauptfläche entsprechen. Zusätzlich wird angegeben ob dieser Punkt auf der Schachtel-Zeichenfläche bereits existiert. Falls es auf einer Fläche nicht möglich seien sollte entsprechende Koordinaten zu berechnen, wird der Zeichenvorgang abgebrochen.\\

Falls in den berechneten Schachtel-Koordinaten Konflikte aufgetreten sind, also Punkte auf den Schachtelflächen bereits vorhanden sind, und das Überschreiben aktiviert ist, werden die bereits vorhandenen Punkte und ihre entsprechenden Punkte auf den anderen Zeichenflächen gelöscht. Ein auf der Hauptzeichenfläche bereits vorhandener Punkt wird ebenso behandelt. Sollte Überschreiben nicht aktiviert sein muss der Zeichenvorgang bei vorhandenen Konflikten abgebrochen werden.\\

Nachdem nun mögliche Konflikte behoben sind können alle berechneten Punkte gezeichnet werden.\\


%%%%%%%%%%%%%%%%%%%%%%%%%%%%%%%%%%%%%%%%%%%%%%%%%%%%%%%%%%%%%%%%%%%%%%%%%%%%%%%
% SCHACHTEL KOORDINATEN %%%%%%%%%%%%%%%%%%%%%%%%%%%%%%%%%%%%%%%%%%%%%%%%%%%%%%%
%%%%%%%%%%%%%%%%%%%%%%%%%%%%%%%%%%%%%%%%%%%%%%%%%%%%%%%%%%%%%%%%%%%%%%%%%%%%%%%
\subsubsection{Berechnen der Schachtel-Koordinaten}
\label{subsubsec:schachtelkoordinaten}

In der Funktion prepare(x, y) der BoxCanvas-Objekte wird von den x-, y-Koordinaten der Hauptzeichenfläche ausgehend die Koordinaten des entsprechenden Punktes der Schachtel-Zeichenfläche berechnet.\\

Ausgehend vom Startpunkt der Schachtelfläche wird eine Verschiebung der Koordinaten addiert. Falls z. B. der Startpunkt der Hauptfläche $(100|100)$ ist und der Startpunkt der Schachtelfläche $(200|300)$, so ergibt sich $(x+100|y+200)$.\\

Zu jeder Schachtelfläche gehört eine Liste traversed\_edges, in der die im aktuellen Zeichenvorgang überquerten Kanten gespeichert werden. Nun wird nacheinander die traverse-Funktion der Kanten für die x-, y-Koordinaten aufgerufen.\\

Nun muss festgestellt werden, ob der berechnete Punkt wiederum außerhalb der Schachtel liegt, ob also eine weitere Kante überquert wurde, oder ob der Punkt innerhalb der Schachtel liegt. Dazu werden Orientierungstest mit den Eckpunkten der beiden Rechtecke, die die Schachtelfläche bilden ausgeführt (Funktion is\_inside(x, y)).\\

Falls der Punkt außerhalb liegt und im aktuellen Zeichenvorgang bereits ein Punkt gezeichnet wurde, also ein gültiger Referenzpunkt vorliegt, wird die zwischen den beiden Punkten liegende Kante festgestellt. Für diese Kante wird ebenfalls die traverse-Funktion aufgerufen, wodurch wir wiederum neue Koordinaten erhalten.\\

Falls kein Referenzpunkt vorhanden ist, kann kein gültiger Punkt berechnet werden.\\

Nun haben wir also gültige Koordinaten für einen auf der Schachtel liegenden Punkt und möglicherweise eine neu überquerte Kante (sollte der Punkt schließlich gezeichnet werden, wird diese Kante zur Liste der Überquerten hinzugefügt). Jetzt wird noch geprüft ob der berechnete Punkt bereits gezeichnet wurde. Diese Information wird zusammen mit den berechneten Koordinaten zurückgegeben.\\


%%%%%%%%%%%%%%%%%%%%%%%%%%%%%%%%%%%%%%%%%%%%%%%%%%%%%%%%%%%%%%%%%%%%%%%%%%%%%%%
% AUTOMATISCHES AUFFÜLLEN %%%%%%%%%%%%%%%%%%%%%%%%%%%%%%%%%%%%%%%%%%%%%%%%%%%%%
%%%%%%%%%%%%%%%%%%%%%%%%%%%%%%%%%%%%%%%%%%%%%%%%%%%%%%%%%%%%%%%%%%%%%%%%%%%%%%%
\subsubsection{Automatisches Auffüllen}
\label{subsubsec:auffuellen}

Falls auf einer Schachtelfläche ein Punkt gelöscht wird (z. B. aufgrund von Überschreibung), wird dieser Löschvorgang gespeichert. Nachdem der aktuelle Punkt gezeichnet wurde, werden die gelöschten Punkt überprüft. Falls ein Punkt tatsächlich gelöscht wurde, also kein neuer Punkt an der gleichen Stelle gezeichnet wurde, wird in der Funktion get\_autofill geprüft, ob auf der entsprechenden Zeichenfläche in der näheren Umgebung bereits gezeichnet wurde. Die Anzahl der überprüften Pixel ist abhängig von der eingestellten Zeichenbreite.\\

Falls ein entsprechender Punkt gefunden wird, wird von diesem aus ``aufgefüllt''. Der gelöschte Punkt wird also so gezeichnet, als wäre er im gleichen Zeichenvorgang entstanden wie der bereits vorhandene Punkt.\\ 


%%%%%%%%%%%%%%%%%%%%%%%%%%%%%%%%%%%%%%%%%%%%%%%%%%%%%%%%%%%%%%%%%%%%%%%%%%%%%%%
% STARTPUNKTE %%%%%%%%%%%%%%%%%%%%%%%%%%%%%%%%%%%%%%%%%%%%%%%%%%%%%%%%%%%%%%%%%
%%%%%%%%%%%%%%%%%%%%%%%%%%%%%%%%%%%%%%%%%%%%%%%%%%%%%%%%%%%%%%%%%%%%%%%%%%%%%%%
\subsubsection{startpunkte (startpoint\_dialog, startpoint\_main)}
\label{subsubsec:startpunkte}

Startpunkte auf Boxen im startpoint\_dialog mit Mausklick oder in
Spinnbox, box\_canvas.offset wird angepasst Startpunkte auf drawing\_canvas
mit mausklick auf Zeichenfläche in startpoint\_main. startpunkt wird
von allen box\_canvas.offset's subtrahiert


%%%%%%%%%%%%%%%%%%%%%%%%%%%%%%%%%%%%%%%%%%%%%%%%%%%%%%%%%%%%%%%%%%%%%%%%%%%%%%%
% UNDO/REDO %%%%%%%%%%%%%%%%%%%%%%%%%%%%%%%%%%%%%%%%%%%%%%%%%%%%%%%%%%%%%%%%%%%
%%%%%%%%%%%%%%%%%%%%%%%%%%%%%%%%%%%%%%%%%%%%%%%%%%%%%%%%%%%%%%%%%%%%%%%%%%%%%%%
\subsubsection{undo/redo (undoable)}
\label{subsubsec:undoRedo}

bei draw werden sich alle punkte x,y gespeichrt, wenn vorher gezeichneter punkt gelöscht wird, werden alle dazugehörigen pixel auf den Boxen und der Zeichenfläche gespeichert. zusätzlich wird der anfangsstatus gespeichet. mit overide, autofill, countinue\_draw und anfangs überquerten kanten auf den Boxen bei undo werden alle diese punkte mit drawing\_canvas.erase
gelöscht und in der redoliste gespeichert. anschließend werden die
durch diese gelöschten pixel vorher überschriebenen pixel wiederhergestellt\\

bei redo wird ein erneutes zeichnen auf der zeichenfläche mit hilfe
von my\_event simuliert\\


%%%%%%%%%%%%%%%%%%%%%%%%%%%%%%%%%%%%%%%%%%%%%%%%%%%%%%%%%%%%%%%%%%%%%%%%%%%%%%%
% SPEICHERN/LADEN %%%%%%%%%%%%%%%%%%%%%%%%%%%%%%%%%%%%%%%%%%%%%%%%%%%%%%%%%%%%%
%%%%%%%%%%%%%%%%%%%%%%%%%%%%%%%%%%%%%%%%%%%%%%%%%%%%%%%%%%%%%%%%%%%%%%%%%%%%%%%
\subsubsection{Speichern/Laden (save\_load)}
\label{subsubsec:speichernLaden}

eine modifzierte Undo-Liste wird abgespeichert als CommonUnfoldFile
{[}{*}.cuf{]}


%%%%%%%%%%%%%%%%%%%%%%%%%%%%%%%%%%%%%%%%%%%%%%%%%%%%%%%%%%%%%%%%%%%%%%%%%%%%%%%
% COMMONUNFOLDFILE %%%%%%%%%%%%%%%%%%%%%%%%%%%%%%%%%%%%%%%%%%%%%%%%%%%%%%%%%%%%
%%%%%%%%%%%%%%%%%%%%%%%%%%%%%%%%%%%%%%%%%%%%%%%%%%%%%%%%%%%%%%%%%%%%%%%%%%%%%%%
\subsubsection{CommonUnfoldFile}
\label{subsubsec:commonunfoldfile}

\begin{itemize}
\item int: Anzahl der Boxen Quadtrupel für jede box: (höhe,breite,tiefe,rotation)
tupel für jede box: boxstartpunkt(x,y) tupel: Zeichenflächenstartpunkt
(x,y) int: Anzahl der gezeichneten Linien (Linie endet nach Mausrelease)
für jede linie:\{ int: anzahl der Pixel Boolean: war overwrite aktiv
Boolean: war autofill aktiv Boolean: war continu line aktiv Liste
von Kanten für jede Box als Start boxes.traversed tripel für jeden
Pixel der linie: (x, y, farbe) \}
\item beim laden werden erst die boxen erstellt, dann die startpunkte gesetzt
und dann das zeichnen simuliert
\end{itemize}


%%%%%%%%%%%%%%%%%%%%%%%%%%%%%%%%%%%%%%%%%%%%%%%%%%%%%%%%%%%%%%%%%%%%%%%%%%%%%%%
% CURSOR %%%%%%%%%%%%%%%%%%%%%%%%%%%%%%%%%%%%%%%%%%%%%%%%%%%%%%%%%%%%%%%%%%%%%%
%%%%%%%%%%%%%%%%%%%%%%%%%%%%%%%%%%%%%%%%%%%%%%%%%%%%%%%%%%%%%%%%%%%%%%%%%%%%%%%
\subsubsection{cursor auf den anderen zeichenflächen anzeigen (cursors)}
\label{subsubsec:cursor}

\begin{itemize}
\item bind von motion auf der Zeichenfläche und den Boxen auf der zeichenfläche:
kreuz auf der box mit offset erstellen auf einer box: eigen offset
von aktueller position abziehen und auf zeichenfläche kreuz erstellen,
dann für alle anderen boxen mit methode von zeichenfläche kreuze berechnen
\end{itemize}


%%%%%%%%%%%%%%%%%%%%%%%%%%%%%%%%%%%%%%%%%%%%%%%%%%%%%%%%%%%%%%%%%%%%%%%%%%%%%%%
% ZEICHNEN AUF BOXEN %%%%%%%%%%%%%%%%%%%%%%%%%%%%%%%%%%%%%%%%%%%%%%%%%%%%%%%%%%
%%%%%%%%%%%%%%%%%%%%%%%%%%%%%%%%%%%%%%%%%%%%%%%%%%%%%%%%%%%%%%%%%%%%%%%%%%%%%%%
\subsubsection{zeichnen auf den Boxen (draw\_from\_box)}
\label{subsubsec:zeichenAufBoxen}

\begin{itemize}
\item rechnet position auf zeichenfläche aus und ruft mit dieser die standart
drawfunktion auf
\end{itemize}


%%%%%%%%%%%%%%%%%%%%%%%%%%%%%%%%%%%%%%%%%%%%%%%%%%%%%%%%%%%%%%%%%%%%%%%%%%%%%%%
% ZEICHENDICKE UND FORM %%%%%%%%%%%%%%%%%%%%%%%%%%%%%%%%%%%%%%%%%%%%%%%%%%%%%%%
%%%%%%%%%%%%%%%%%%%%%%%%%%%%%%%%%%%%%%%%%%%%%%%%%%%%%%%%%%%%%%%%%%%%%%%%%%%%%%%
\subsubsection{zeichendicke und form (choose\_shape, control\_panel, draw\_with,
drawing\_canvas.width\_draw)}
\label{subsubsec:zeichendickeForm}

\begin{itemize}
\item für jeden pixel der fläche wird draw aufgerufen
\end{itemize}


%%%%%%%%%%%%%%%%%%%%%%%%%%%%%%%%%%%%%%%%%%%%%%%%%%%%%%%%%%%%%%%%%%%%%%%%%%%%%%%
% GENERIEUNG VON SCHACHTELN UND PREVIEW %%%%%%%%%%%%%%%%%%%%%%%%%%%%%%%%%%%%%%%
%%%%%%%%%%%%%%%%%%%%%%%%%%%%%%%%%%%%%%%%%%%%%%%%%%%%%%%%%%%%%%%%%%%%%%%%%%%%%%%
\subsubsection{Generierung der Schachteln und Preview}
\label{subsubsec:generierungPreview}
Muss noch text her! XXX TODO


%%%%%%%%%%%%%%%%%%%%%%%%%%%%%%%%%%%%%%%%%%%%%%%%%%%%%%%%%%%%%%%%%%%%%%%%%%%%%%%
% ROTATION UND GRID %%%%%%%%%%%%%%%%%%%%%%%%%%%%%%%%%%%%%%%%%%%%%%%%%%%%%%%%%%%
%%%%%%%%%%%%%%%%%%%%%%%%%%%%%%%%%%%%%%%%%%%%%%%%%%%%%%%%%%%%%%%%%%%%%%%%%%%%%%%
\subsubsection{Rotation und Grid}
\label{subsubsec:rotationGrid}
Text XXX TODO