\section{Softwarearchitektur}
\label{sec:backend}

In diesem Teil der Arbeit wird die technische Umsetzung des Projekts erläutert. Dazu gehören die funktionalen Anforderungen, welche zum größten Teil durch unsere Aufgabenstellung gegeben waren, sowie die technische Beschreibung der Funktionsweise des Programms. Auf Schwierigkeiten und Herausforderungen bezüglich der technischen Umsetzung und Algorithmen wird näher eingegangen.\\


%%%%%%%%%%%%%%%%%%%%%%%%%%%%%%%%%%%%%%%%%%%%%%%%%%%%%%%%%%%%%%%%%%%%%%%%%%%%%%%
%%%%%%%%%%%%%%%%%%%%%%%%%%%%%%%%%%%%%%%%%%%%%%%%%%%%%%%%%%%%%%%%%%%%%%%%%%%%%%%
%%%%%%%%%%%%%%%%%%%%%%%%%%%%%%%%%%%%%%%%%%%%%%%%%%%%%%%%%%%%%%%%%%%%%%%%%%%%%%%
% FUNKTIONALE ANFORDERUNGEN %%%%%%%%%%%%%%%%%%%%%%%%%%%%%%%%%%%%%%%%%%%%%%%%%%%
%%%%%%%%%%%%%%%%%%%%%%%%%%%%%%%%%%%%%%%%%%%%%%%%%%%%%%%%%%%%%%%%%%%%%%%%%%%%%%%
\subsection{Funktionale Anforderungen}
\label{subsec:anforderungen}

Im Folgenden werden wir die funktionalen Anforderungen an unser Programm erläutern. Diese waren vor allem durch die Aufgabenstellung gegeben. Ziel war es eine grafische Oberfläche zu entwickeln mit der es möglich ist das Konzept des simultanen Zeichnens zu verwenden.

\paragraph{Optionale Verknüpfung von Randstücken}
Wird ein Gitternetz eines Quaders gezeichnet, so entstehen doppelte Kanten, obwohl es diese Kanten in der Realität nur einmal gibt. Hier galt es eine Regel zu finden um Ungenauigkeiten zu vermeiden.

\paragraph{Unterstützung der Anfangszuordnung}
Bevor mit dem Zeichnen begonnen wird, sollen die Startpunkte auf den einzelnen Schachteln ausgewählt werden können.

\paragraph{Automatisierung der Verarbeitung verschiedener Schachteln}
Es soll durch den Benutzer ausgewählt werden, welche Schachteln er bearbeiten möchte.

\paragraph{Automatisches Auffüllen}
Wird an einer Stelle etwas weggenommen, weil es zum Beispiel durch eine andere Fläche übermalt wurde, so sollen automatisch die Zwischenräume mit der passenden Farbe aufgefüllt werden.

%%%%%%%%%%%%%%%%%%%%%%%%%%%%%%%%%%%%%%%%%%%%%%%%%%%%%%%%%%%%%%%%%%%%%%%%%%%%%%%
%%%%%%%%%%%%%%%%%%%%%%%%%%%%%%%%%%%%%%%%%%%%%%%%%%%%%%%%%%%%%%%%%%%%%%%%%%%%%%%
%%%%%%%%%%%%%%%%%%%%%%%%%%%%%%%%%%%%%%%%%%%%%%%%%%%%%%%%%%%%%%%%%%%%%%%%%%%%%%%
% WEITERES AN FUNKTIONALITÄT %%%%%%%%%%%%%%%%%%%%%%%%%%%%%%%%%%%%%%%%%%%%%%%%%%
%%%%%%%%%%%%%%%%%%%%%%%%%%%%%%%%%%%%%%%%%%%%%%%%%%%%%%%%%%%%%%%%%%%%%%%%%%%%%%%
\subsection{Weitere Funktionen}
\label{subsec:weiteres}

\begin{itemize}
  \item Zeichnen soll auch direkt auf den Schachteln möglich sein
  \item Anzeigen der Cursorposition auf allen Schachteln
  \item Anzeigen eines Koordinatensystems (Grid als Hilfslinien)
  \item Undo/Redo unterstützen 
  \item Speichern/Ladenbereitstellen
  \item Zeichendicke und Form des Pinsels soll beim Zeichnen auswählbar sein
  \item Rotation der Schachteln
\end{itemize}

%%%%%%%%%%%%%%%%%%%%%%%%%%%%%%%%%%%%%%%%%%%%%%%%%%%%%%%%%%%%%%%%%%%%%%%%%%%%%%%
%%%%%%%%%%%%%%%%%%%%%%%%%%%%%%%%%%%%%%%%%%%%%%%%%%%%%%%%%%%%%%%%%%%%%%%%%%%%%%%
%%%%%%%%%%%%%%%%%%%%%%%%%%%%%%%%%%%%%%%%%%%%%%%%%%%%%%%%%%%%%%%%%%%%%%%%%%%%%%%
% FUNKTIONSWEISE %%%%%%%%%%%%%%%%%%%%%%%%%%%%%%%%%%%%%%%%%%%%%%%%%%%%%%%%%%%%%%
%%%%%%%%%%%%%%%%%%%%%%%%%%%%%%%%%%%%%%%%%%%%%%%%%%%%%%%%%%%%%%%%%%%%%%%%%%%%%%%
\subsection{Technischen Umsetzung}
\label{subsec:funktionsweise}

Alle Zeichenvorgänge gehen von einer Benutzereingabe auf der Hauptzeichenfläche aus, auch beim Zeichnen auf den Schachtel-Zeichenflächen wird ein entsprechendes Ereignis auf der Hauptfläche ausgelöst, welches in der \texttt{draw}-Methode des \texttt{DrawingCanvas}-Objektes behandelt wird.

Von den $x$- und $y$-Koordinaten dieses Ereignisses ausgehend wird von jeder Schachtel-Zeichenfläche die Funktion \texttt{prepare(x,y)} aufgerufen.

In den Zeichenflächen werden die Koordinaten berechnet, die beim aktuellen Zeichenvorgang dem Punkt auf der Hauptfläche entsprechen. Zusätzlich wird angegeben, ob dieser Punkt auf der Schachtel-Zeichenfläche bereits existiert. Falls es auf einer Fläche nicht möglich sein sollte, entsprechende Koordinaten zu berechnen, wird der Zeichenvorgang abgebrochen.\\

Falls in den berechneten Schachtel-Koordinaten Konflikte aufgetreten sind, also Punkte auf den Schachtelflächen bereits vorhanden sind, und das \emph{Überschreiben} (\texttt{Override}) aktiviert ist, werden die bereits vorhandenen Punkte und ihre entsprechenden Punkte auf den anderen Zeichenflächen gelöscht. Ein auf der Hauptzeichenfläche bereits vorhandener Punkt wird ebenso behandelt. Sollte \emph{Überschreiben} nicht aktiviert sein, muss der Zeichenvorgang bei vorhandenen Konflikten abgebrochen werden.

Nachdem nun mögliche Konflikte behoben sind, können alle berechneten Punkte gezeichnet werden.


%%%%%%%%%%%%%%%%%%%%%%%%%%%%%%%%%%%%%%%%%%%%%%%%%%%%%%%%%%%%%%%%%%%%%%%%%%%%%%%
% SCHACHTEL KOORDINATEN %%%%%%%%%%%%%%%%%%%%%%%%%%%%%%%%%%%%%%%%%%%%%%%%%%%%%%%
%%%%%%%%%%%%%%%%%%%%%%%%%%%%%%%%%%%%%%%%%%%%%%%%%%%%%%%%%%%%%%%%%%%%%%%%%%%%%%%
\subsubsection{Berechnen der Schachtel-Koordinaten}
\label{subsubsec:schachtelkoordinaten}

In der Funktion \texttt{prepare(x,y)} der \texttt{BoxCanvas}-Objekte wird von den $x$-, $y$-Koordinaten der Hauptzeichenfläche ausgehend die Koordinaten des entsprechenden Punktes der Schachtel-Zeichenfläche berechnet.

Ausgehend vom Startpunkt der Schachtelfläche wird eine Verschiebung der Koordinaten addiert. Falls z. B. der Startpunkt der Hauptfläche $(100|100)$ ist und der Startpunkt der Schachtelfläche $(200|300)$, so ergibt sich $(x+100|y+200)$.

Zu jeder Schachtelfläche gehört eine Liste \texttt{traversed\_edges}, in der die im aktuellen Zeichenvorgang überquerten Kanten gespeichert werden. Nun wird nacheinander die \texttt{traverse}-Funktion der Kanten für die $x$-, $y$-Koordinaten aufgerufen.

Nun muss festgestellt werden, ob der berechnete Punkt wiederum außerhalb der Schachtel liegt, ob also eine weitere Kante überquert wurde, oder ob der Punkt innerhalb der Schachtel liegt. Dazu werden Orientierungstests mit den Eckpunkten der beiden Rechtecke, die die Schachtelfläche bilden ausgeführt (Funktion \texttt{is\_inside(x,y)}).

Falls der Punkt außerhalb liegt und im aktuellen Zeichenvorgang bereits ein Punkt gezeichnet wurde, also ein gültiger Referenzpunkt vorliegt, wird die zwischen den beiden Punkten liegende Kante festgestellt. Für diese Kante wird ebenfalls die \texttt{traverse}-Funktion aufgerufen, wodurch wir wiederum neue Koordinaten erhalten.

Falls kein Referenzpunkt vorhanden ist, kann kein gültiger Punkt berechnet werden.\\

Nun haben wir also gültige Koordinaten für einen auf der Schachtel liegenden Punkt und möglicherweise eine neu überquerte Kante (sollte der Punkt schließlich gezeichnet werden, wird diese Kante zur Liste der überquerten Kanten hinzugefügt). Jetzt wird noch geprüft ob der berechnete Punkt bereits gezeichnet wurde. Diese Information wird zusammen mit den berechneten Koordinaten zurückgegeben.\\


%%%%%%%%%%%%%%%%%%%%%%%%%%%%%%%%%%%%%%%%%%%%%%%%%%%%%%%%%%%%%%%%%%%%%%%%%%%%%%%
% AUTOMATISCHES AUFFÜLLEN %%%%%%%%%%%%%%%%%%%%%%%%%%%%%%%%%%%%%%%%%%%%%%%%%%%%%
%%%%%%%%%%%%%%%%%%%%%%%%%%%%%%%%%%%%%%%%%%%%%%%%%%%%%%%%%%%%%%%%%%%%%%%%%%%%%%%
\subsubsection{Automatisches Auffüllen}
\label{subsubsec:auffuellen}

Falls auf einer Schachtelfläche ein Punkt gelöscht wird (\zB aufgrund von Überschreibung, \texttt{Override}), wird dieser Löschvorgang gespeichert. Nachdem der aktuelle Punkt gezeichnet wurde, werden die gelöschten Punkt überprüft. Falls ein Punkt tatsächlich gelöscht wurde, also kein neuer Punkt an der gleichen Stelle gezeichnet wurde, wird in der Funktion \texttt{get\_autofill} geprüft, ob auf der entsprechenden Zeichenfläche in der näheren Umgebung bereits gezeichnet wurde. Die Anzahl der überprüften Pixel ist abhängig von der eingestellten Zeichenbreite.

Falls ein entsprechender Punkt gefunden wird, wird von diesem aus "`aufgefüllt"'. Der gelöschte Punkt wird also so gezeichnet, als wäre er im gleichen Zeichenvorgang entstanden wie der bereits vorhandene Punkt.


%%%%%%%%%%%%%%%%%%%%%%%%%%%%%%%%%%%%%%%%%%%%%%%%%%%%%%%%%%%%%%%%%%%%%%%%%%%%%%%
% STARTPUNKTE %%%%%%%%%%%%%%%%%%%%%%%%%%%%%%%%%%%%%%%%%%%%%%%%%%%%%%%%%%%%%%%%%
%%%%%%%%%%%%%%%%%%%%%%%%%%%%%%%%%%%%%%%%%%%%%%%%%%%%%%%%%%%%%%%%%%%%%%%%%%%%%%%
\subsubsection{Setzen von Startpunkten}
\label{subsubsec:startpunkte}
Startpunkte werden für jede Schachtelfläche gesetzt. Diese Werte werden als Offset in \texttt{box\_canvas.offset} gespeichert. Wird der Startpunkt auf der Zeichenfläche gesetzt, so wird dieser von allen \texttt{box\_canvas.offset}`s subtrahiert.


%%%%%%%%%%%%%%%%%%%%%%%%%%%%%%%%%%%%%%%%%%%%%%%%%%%%%%%%%%%%%%%%%%%%%%%%%%%%%%%
% UNDO/REDO %%%%%%%%%%%%%%%%%%%%%%%%%%%%%%%%%%%%%%%%%%%%%%%%%%%%%%%%%%%%%%%%%%%
%%%%%%%%%%%%%%%%%%%%%%%%%%%%%%%%%%%%%%%%%%%%%%%%%%%%%%%%%%%%%%%%%%%%%%%%%%%%%%%
\subsubsection{Undo und Redo}
\label{subsubsec:undoRedo}

Beim Zeichenvorgang werden alle Punkte in der Form $(x,y)$ gespeichert. Wenn ein vorher gezeichneter Punkt gelöscht wird, so werden alle dazugehörigen Punkte auf den einzelnen Boxen und der Zeichenfläche gespeichert. Zusätzlich wird der Anfangsstatus gespeichert. Dazu gehören die Zustände: \texttt{overwrite}, \texttt{autofill}, \texttt{continue} und die anfangs überquerten Kanten auf den Schachtelflächen.\\

Bei \texttt{Undo} werden alle diese Punkte mit \texttt{drawing\_canvas.erase} gelöscht und in der Redo-Liste gespeichert. Anschließend werden die durch diese gelöschten Punkt vorher überschriebenen Punkte wiederhergestellt\\

Bei \texttt{Redo} wird ein erneutes Zeichnen auf der Zeichenfläche mit Hilfe
von \texttt{my\_event} simuliert.\\


%%%%%%%%%%%%%%%%%%%%%%%%%%%%%%%%%%%%%%%%%%%%%%%%%%%%%%%%%%%%%%%%%%%%%%%%%%%%%%%
% SPEICHERN/LADEN %%%%%%%%%%%%%%%%%%%%%%%%%%%%%%%%%%%%%%%%%%%%%%%%%%%%%%%%%%%%%
%%%%%%%%%%%%%%%%%%%%%%%%%%%%%%%%%%%%%%%%%%%%%%%%%%%%%%%%%%%%%%%%%%%%%%%%%%%%%%%
\subsubsection{Speichern und Laden}
\label{subsubsec:speichernLaden}
Wird der Status einer Zeichenfläche abgespeichert, so wird eine modifzierte Undo-Liste als CommonUnfoldFile (\texttt{*.cuf}) abgespeichert.\\

Struktur des CommonUnfoldFiles:

\begin{itemize}
\item Integer: Anzahl der Schachteln
\item Quadtrupel für jede Schachtel: (Höhe, Breite, Tiefe, Rotation)
\item Tupel für jede Schachtel: Startpunkt der Schachtel $(x,y)$
\item Tupel: Startpunkt der Zeichenfläche $(x,y)$
\item Integer: Anzahl der gezeichneten Linien (Linie endet nach dem Loslassen der Maus)
\item Für jede Linie:
  \begin{itemize}
    \item Integer: Anzahl der Pixel
    \item Boolean: war der Status \texttt{overwrite} aktiv
    \item Boolean: war der Status \texttt{autofill} aktiv
    \item Boolean: war der Status \texttt{continue} aktiv
    \item Liste von Kanten für jede Schachtel (als Start \texttt{boxes.traversed})
    \item Tripel für jeden Pixel der Linie: $(x, y, Farbe)$
  \end{itemize}
\end{itemize}

Beim Laden einer \texttt{*.cuf} Datei, werden erst die Schachteln erstellt, dann die Startpunkte gesetzt und im Anschluss das Zeichnen simuliert.


%%%%%%%%%%%%%%%%%%%%%%%%%%%%%%%%%%%%%%%%%%%%%%%%%%%%%%%%%%%%%%%%%%%%%%%%%%%%%%%
% CURSOR %%%%%%%%%%%%%%%%%%%%%%%%%%%%%%%%%%%%%%%%%%%%%%%%%%%%%%%%%%%%%%%%%%%%%%
%%%%%%%%%%%%%%%%%%%%%%%%%%%%%%%%%%%%%%%%%%%%%%%%%%%%%%%%%%%%%%%%%%%%%%%%%%%%%%%
\subsubsection{Cursor}
\label{subsubsec:cursor}

Zeichnet man auf der Zeichenfläche wird gleichzeitig auf allen Schachteln die Postion der Maus mit Hilfe eines kleinen Kreuzes angezeigt. Befindet man sich mit der Maus auf einer Schachtel so wird auch diese Position auf allen anderen Schachteln und der Zeichenfläche angezeigt.\\

(1) Befindet man sich mit der Maus auf der Zeichenfläche so werden die Kreuze auf den Schachteln mit Hilfe eines Offsets berechnet und erstellt.

(2) Befindet man sich auf einer Schachtel, so wird der eigene Offset von der aktuellen Position abgezogen und das Kreuz auf der Zeichenfläche erstellt. Alle weiteren Kreuze auf den anderen Schachteln werden nun mit Methode (1) errechnet.


%%%%%%%%%%%%%%%%%%%%%%%%%%%%%%%%%%%%%%%%%%%%%%%%%%%%%%%%%%%%%%%%%%%%%%%%%%%%%%%
% CURSOR %%%%%%%%%%%%%%%%%%%%%%%%%%%%%%%%%%%%%%%%%%%%%%%%%%%%%%%%%%%%%%%%%%%%%%
%%%%%%%%%%%%%%%%%%%%%%%%%%%%%%%%%%%%%%%%%%%%%%%%%%%%%%%%%%%%%%%%%%%%%%%%%%%%%%%
\subsubsection{Zeichnen auf Schachteln}
\label{subsubsec:zeichnenaufbox}

Zusätzlich zum normalen Zeichen auf der Zeichenfläche kann auf den einzelnen Schachteln auch gezeichnet werden. Die Position auf der Schachtel wird auf die Position auf der Zeichenfläche umgerechnet und dort wird dann die standardmäßige \texttt{Draw}-Funktion aufgerufen.


%%%%%%%%%%%%%%%%%%%%%%%%%%%%%%%%%%%%%%%%%%%%%%%%%%%%%%%%%%%%%%%%%%%%%%%%%%%%%%%
% ZEICHENDICKE UND FORM %%%%%%%%%%%%%%%%%%%%%%%%%%%%%%%%%%%%%%%%%%%%%%%%%%%%%%%
%%%%%%%%%%%%%%%%%%%%%%%%%%%%%%%%%%%%%%%%%%%%%%%%%%%%%%%%%%%%%%%%%%%%%%%%%%%%%%%
\subsubsection{Zeichendicke und Form}
\label{subsubsec:zeichendickeForm}

Rund um die Position des Cursors wird je nach gewählter Form und Größe jeder Pixel der entsprechenden Fläche berechnet und mit \texttt{draw} gezeichnet (\texttt{draw\_width}).


%%%%%%%%%%%%%%%%%%%%%%%%%%%%%%%%%%%%%%%%%%%%%%%%%%%%%%%%%%%%%%%%%%%%%%%%%%%%%%%
% GENERIEUNG VON SCHACHTELN UND PREVIEW %%%%%%%%%%%%%%%%%%%%%%%%%%%%%%%%%%%%%%%
%%%%%%%%%%%%%%%%%%%%%%%%%%%%%%%%%%%%%%%%%%%%%%%%%%%%%%%%%%%%%%%%%%%%%%%%%%%%%%%
\subsubsection{Generierung von Schachteln}
\label{subsubsec:generierungPreview}
Schachteln werden immer über den Flächeninhalt erzeugt. Ist die Dimension einer Schachtel gegeben, so wird im ersten Schritt der Oberflächeninhal (\texttt{calc\_surface\_area}) berechnet. Im zweiten Schritt werden alle Schachteln $(a\times b \times c)$ berechnet, welche den gleichen Oberflächeninhalt haben.\\


\begin{lstlisting}[caption=Generierung von Schachteln]
# Calculates the surfaceare by edgelenght a,b,c
def calc_surface_area (a, b, c):
  return 2 * a * b + 2 * a * c + 2 * b * c

# Input: surfacearea of a cuboid
# Returs: all tupel (a,b,c), where a,b,c are the edgelenght beeing in that same equivalence relation
def create_triple(surface):
  surface = surface/2
  tripelList = []
  for a in range(1, int(sqrt(surface // 3)) + 1):
    b = a
    c, rem = divmod(surface - a * b, a + b)
    while (c >= b):
      if rem == 0:
        tripelList.append((a, b, c, 0)) 
      b = b + 1
      c, rem = divmod(surface - a * b, a + b)
  
  return tripelList
\end{lstlisting}