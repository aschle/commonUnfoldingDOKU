\section{Zusammenfassung und Ausblick}
\label{sec:zusammenfassung}
Im nun letzten Kapitel erfolgt eine Einschätzung der Egebnisse des Softwarepraktikums und des Gelernten. Als Abschluss wird ein Ausblick auf weitere Arbeiten gegeben.

%%%%%%%%%%%%%%%%%%%%%%%%%%%%%%%%%%%%%%%%%%%%%%%%%%%%%%%%%%%%%%%%%%%%%%%%%%%%%%%
%%%%%%%%%%%%%%%%%%%%%%%%%%%%%%%%%%%%%%%%%%%%%%%%%%%%%%%%%%%%%%%%%%%%%%%%%%%%%%%
%%%%%%%%%%%%%%%%%%%%%%%%%%%%%%%%%%%%%%%%%%%%%%%%%%%%%%%%%%%%%%%%%%%%%%%%%%%%%%%
% ERGEBNISSE DES SOFTWAREPROJEKTES %%%%%%%%%%%%%%%%%%%%%%%%%%%%%%%%%%%%%%%%%%%%
%%%%%%%%%%%%%%%%%%%%%%%%%%%%%%%%%%%%%%%%%%%%%%%%%%%%%%%%%%%%%%%%%%%%%%%%%%%%%%%
\subsection{Ergebnisse des Softwareprojekts}
\label{subsec:ergebnisse}

Im Großen und Ganzen haben wir unser Endziel des Softwarepraktikums erreicht und erfolgreich abgeschlossen. Dabei haben wir viel über Teamarbeit und Koordination von Softwareprozessen gelernt. Getroffene Designentscheideungen, wie beispielsweise die Wahl der Programmiersprach lässt sich im laufenden Prozess nicht mehr verändern, da hätten wir vorber bessere Entscheidungskritereien aufstellen sollen. Da unser Projekt keinen großen algoritmischen Anteil enthielt, war die Implementierung keine Herausforderung, bei unserem Projekt hätte man beispielsweise den Schwerpunkt auf Interfacegestaltung legen können anstelle auf Algorithmen.

%%%%%%%%%%%%%%%%%%%%%%%%%%%%%%%%%%%%%%%%%%%%%%%%%%%%%%%%%%%%%%%%%%%%%%%%%%%%%%%
%%%%%%%%%%%%%%%%%%%%%%%%%%%%%%%%%%%%%%%%%%%%%%%%%%%%%%%%%%%%%%%%%%%%%%%%%%%%%%%
%%%%%%%%%%%%%%%%%%%%%%%%%%%%%%%%%%%%%%%%%%%%%%%%%%%%%%%%%%%%%%%%%%%%%%%%%%%%%%%
% ELESSONS LEARNED %%%%%%%%%%%%%%%%%%%%%%%%%%%%%%%%%%%%%%%%%%%%%%%%%%%%%%%%%%%%
%%%%%%%%%%%%%%%%%%%%%%%%%%%%%%%%%%%%%%%%%%%%%%%%%%%%%%%%%%%%%%%%%%%%%%%%%%%%%%%
\subsection{Lessons Learned}
\label{subsec:lessons}

Kurz vor Ende des Projektes mussten wir feststellen, das eine wichtige Funktinalität in unserem Projekt komplett fehlte, das automatische Auffüllen von frei werdenden Flächen. Durch eine regelmäßige Kommunikation und Feedbackrunden mit unserem Betreuer hätten wir diesen Stress am Ende selbst vermeiden können. 
Das Abspringen ein Teammitlieds vor Ende des Projekt ist sehr schade, bei einem nächsten Projekt muss darauf geachtet werden, das alle Teammitglieder im Boot bleiben und gleichermaßen am Projekt beteiligt werden. Man sollt vor Prokjektbeginn die Motivation der einzelnen Mitglieder richtig einschätzen können, sodass soetwas nicht passiert.

%%%%%%%%%%%%%%%%%%%%%%%%%%%%%%%%%%%%%%%%%%%%%%%%%%%%%%%%%%%%%%%%%%%%%%%%%%%%%%%
%%%%%%%%%%%%%%%%%%%%%%%%%%%%%%%%%%%%%%%%%%%%%%%%%%%%%%%%%%%%%%%%%%%%%%%%%%%%%%%
%%%%%%%%%%%%%%%%%%%%%%%%%%%%%%%%%%%%%%%%%%%%%%%%%%%%%%%%%%%%%%%%%%%%%%%%%%%%%%%
% AUSBLICK %%%%%%%%%%%%%%%%%%%%%%%%%%%%%%%%%%%%%%%%%%%%%%%%%%%%%%%%%%%%%%%%%%%%
%%%%%%%%%%%%%%%%%%%%%%%%%%%%%%%%%%%%%%%%%%%%%%%%%%%%%%%%%%%%%%%%%%%%%%%%%%%%%%%
\subsection{Ausblick}
\label{subsec:ausblick}
Die Perforamnce des Programmes muss erhöht werden, beim sehr schnellen Zeichnen, werden einige Punkte nicht gezeichnet, was auf das langsame Reagieren der Maus zurückzuführen ist. Eine besser Benutzbare und intuitive Oberfläche sollte geplant werden. Es wäre denkbar einen Algorithmus zu entwickelt, welcher selbständig die Oberfläche ausmalt, solange bis alle Flächen komplett ausgemalt sind.