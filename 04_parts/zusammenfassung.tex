\section{Zusammenfassung und Ausblick}
\label{sec:zusammenfassung}
Im letzten Kapitel erfolgt eine Einschätzung der Ergebnisse des Softwarepraktikums und des Gelernten. Als Abschluss wird ein Ausblick auf weitere Arbeiten gegeben.

%%%%%%%%%%%%%%%%%%%%%%%%%%%%%%%%%%%%%%%%%%%%%%%%%%%%%%%%%%%%%%%%%%%%%%%%%%%%%%%
%%%%%%%%%%%%%%%%%%%%%%%%%%%%%%%%%%%%%%%%%%%%%%%%%%%%%%%%%%%%%%%%%%%%%%%%%%%%%%%
%%%%%%%%%%%%%%%%%%%%%%%%%%%%%%%%%%%%%%%%%%%%%%%%%%%%%%%%%%%%%%%%%%%%%%%%%%%%%%%
% ERGEBNISSE DES SOFTWAREPROJEKTES %%%%%%%%%%%%%%%%%%%%%%%%%%%%%%%%%%%%%%%%%%%%
%%%%%%%%%%%%%%%%%%%%%%%%%%%%%%%%%%%%%%%%%%%%%%%%%%%%%%%%%%%%%%%%%%%%%%%%%%%%%%%
\subsection{Ergebnisse des Softwareprojekts}
\label{subsec:ergebnisse}

Im Großen und Ganzen haben wir unser Endziel des Softwarepraktikums erreicht und erfolgreich abgeschlossen. Dabei haben wir viel über Teamarbeit und die Koordination von Softwareprozessen gelernt. Getroffene Designentscheidungen, wie beispielsweise die Wahl der Programmiersprache, lassen sich oft im laufenden Prozess nicht mehr verändern, möglicherweise hätten wir im Vorfeld bessere Entscheidungskriterien aufstellen sollen. Da unser Projekt keinen großen algorithmischen Anteil enthielt, war die Implementierung keine große Herausforderung. Eine Ausnahme war das Implementieren der \texttt{Autofill}-Funktion. Bei unserem Projekt hätte man den Schwerpunkt mehr auf Interface-Gestaltung legen können anstelle auf Algorithmen.

%%%%%%%%%%%%%%%%%%%%%%%%%%%%%%%%%%%%%%%%%%%%%%%%%%%%%%%%%%%%%%%%%%%%%%%%%%%%%%%
%%%%%%%%%%%%%%%%%%%%%%%%%%%%%%%%%%%%%%%%%%%%%%%%%%%%%%%%%%%%%%%%%%%%%%%%%%%%%%%
%%%%%%%%%%%%%%%%%%%%%%%%%%%%%%%%%%%%%%%%%%%%%%%%%%%%%%%%%%%%%%%%%%%%%%%%%%%%%%%
% ELESSONS LEARNED %%%%%%%%%%%%%%%%%%%%%%%%%%%%%%%%%%%%%%%%%%%%%%%%%%%%%%%%%%%%
%%%%%%%%%%%%%%%%%%%%%%%%%%%%%%%%%%%%%%%%%%%%%%%%%%%%%%%%%%%%%%%%%%%%%%%%%%%%%%%
\subsection{Lessons Learned}
\label{subsec:lessons}

Kurz vor Ende des Projektes mussten wir feststellen, dass eine wichtige Funktionalität in unserem Projekt komplett fehlte: das automatische Auffüllen von frei werdenden Flächen. Durch eine regelmäßige Kommunikation und Feedbackrunden mit unserem Betreuer hätten wir diesen Stress am Ende des Projekts vermeiden können. 
Das Abspringen eines Teammitglieds kurz vor Abschluss des Projekt ist sehr schade, bei einem nächsten Projekt muss darauf geachtet werden, dass alle Teammitglieder im Boot bleiben und gleichermaßen am Projekt beteiligt werden. Man sollte vor Projektbeginn die Motivation der einzelnen Mitglieder richtig einschätzen können, sodass so etwas nicht passiert.

%%%%%%%%%%%%%%%%%%%%%%%%%%%%%%%%%%%%%%%%%%%%%%%%%%%%%%%%%%%%%%%%%%%%%%%%%%%%%%%
%%%%%%%%%%%%%%%%%%%%%%%%%%%%%%%%%%%%%%%%%%%%%%%%%%%%%%%%%%%%%%%%%%%%%%%%%%%%%%%
%%%%%%%%%%%%%%%%%%%%%%%%%%%%%%%%%%%%%%%%%%%%%%%%%%%%%%%%%%%%%%%%%%%%%%%%%%%%%%%
% AUSBLICK %%%%%%%%%%%%%%%%%%%%%%%%%%%%%%%%%%%%%%%%%%%%%%%%%%%%%%%%%%%%%%%%%%%%
%%%%%%%%%%%%%%%%%%%%%%%%%%%%%%%%%%%%%%%%%%%%%%%%%%%%%%%%%%%%%%%%%%%%%%%%%%%%%%%
\subsection{Ausblick}
\label{subsec:ausblick}
Die Performance des Programms muss erhöht werden, da beim schnellen Zeichnen einige Punkte nicht gezeichnet werden. Dies ist auf die langsame Reaktion der Maus-Events im TkInter-Framework zurückzuführen. Eine besser benutzbare und intuitive Oberfläche sollte geplant werden. Es wäre denkbar einen Algorithmus zu entwickeln, welcher selbständig die Oberfläche ausmalt, solange, bis alle Flächen komplett ausgemalt sind.
